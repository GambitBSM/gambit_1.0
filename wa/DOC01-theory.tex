%%%%%%%%%%%%%%%%%%%%%%%%%%%%%%
%%%%%%%%%%%%%%%%%%%%%%%%%%%%%%
\section{Muon yields from annihilation in the Earth/Sun -- theory}

We need to take into account all processes that yield muon neutrinos from
annihilation in the Earth/Sun. To do this, we use a Monte Carlo, WimpSim \cite{wimpsim}, to simulate
annihilations in the center of the Sun/Earth, neutrino oscillations and neutrino interactions on the way out of the Sun/Earth and to the detector.

%%%%%%%%%%%%%%%%%%%%
%%%%%%%%%%%%%%%%%%%%%%%%%%%%%%
\subsection{Monte Carlo simulations with WimpSim}
\label{sec:nt-mcsim}

We need to
evaluate the yield of different particles per neutralino annihilation.
The hadronization and/or decay of the annihilation products are
simulated with {\sc Pythia} \cite{pythia} 6.414
and we here describe how the simulations are done.
For annihilation in the Sun/Earth 
the simulations are done for a set of 22 neutralino
masses, $m_{\chi}$ = 3, 6, 10, 25, 50, 80.3, 91.2, 100, 150, 176, 200, 250,
350, 500, 750, 1000, 1500, 2000, 3000, 5000, 7500 and 10000 GeV\@.
We tabulate the yields and then interpolate these tables in \ds.

We are mainly interested in the flux of high energy muon neutrinos
     and neutrino-induced muons at a neutrino telescope.  We simulate 13
     `fundamental' annihilation channels,      
%     $u\bar{u}$$c\bar{c}$, $b\bar{b}$,
%     $t\bar{t}$, $\tau^+\tau^-$, $W^+W^-$ and $Z^0 Z^{0}$
for each mass
     (where kinematically allowed) above. In Table~\ref{tab:wa-channels} we list the `fundamental' channels for which simulations are run and the full set of more complex channels.
     Pions and kaons get stopped
     before they decay and are thus made stable in the {\sc Pythia}
     simulations so that they don't produce any neutrinos.  For
     annihilation channels containing Higgs bosons, we can calculate
     the yield from these fundamental channels by letting the Higgs
     bosons decaying in flight (see below).  We also take into account
     the energy losses of $B$-mesons in the Sun and the Earth by
     following the approximate treatment of \cite{RS} but with updated
     $B$-meson interaction cross sections as given in
     \cite{joakimthesis}.  We also take neutrino-interactions on the
     way out of the Sun into account by considering the charged-current
     interaction as a neutrino-loss and the neutral current
     interactions are simulated with \code{nusigma} \cite{nusigma}.  The
     neutrino-nucleon charged current interactions close to the
     detector are also simulated with \code{nusigma} and finally the
     multiple Coulomb scattering of the muon on its way to the detector
     is calculated using distributions from~\cite{PDG}. We have used the CTEQ6
     structure functions in these simulations.
     We also take into account neutrino oscillations with a full three-neutrino Monte Carlo. All of these processes are put into the simulation package \code{WimpSim} \code{wimpsim} that can be downloaded separately. Results from simulation runs with this package are included with \ds.
     For more details on these simulations, see~\cite{wimpnu}.
     
\begin{table}
\begin{tabular}{llll}
Annihilation channel & & Internal channel & Internal channel array index  \\
\code{ch} & Particles & \code{chi} & \code{chii} \\ \hline
1 & $S_1^0 S_1^0$ & - & - \\
2 & $S_1^0 S_2^0$ & - & - \\
3 & $S_2^0 S_2^0$ & - & - \\
4 & $S_3^0 S_3^0$ & - & - \\
5 & $S_1^0 S_3^0$ & - & - \\
6 & $S_2^0 S_3^0$ & - & - \\
7 & $S^- S^+$ & - & - \\
8 & $Z^0 S_1^0$ & - & - \\
9 & $Z^0 S_2^0$ & - & - \\
10 & $Z^0 S_3^0$ & - & - \\
11 & $W^- S^+ / W^+ W^-$ & - & - \\
12 & $Z^0 Z^0$ & 9 & 9 \\
13 & $W^+ W^-$ & 8 & 8 \\
14 & $\nu_e \bar{\nu}_e$ & 12 & 11 \\
15 & $e^+ e^-$ & - & - \\
16 & $\nu_\mu \bar{\nu}_\mu$ & 13 & 12 \\
17 & $\mu^+ \mu^-$ & 10 & - \\
18 & $\nu_\tau \bar{\nu}_\tau$ & 14 & 13 \\
19 & $\tau^+ \tau^-$ & 11 & 10 \\
20 & $u \bar{u}$ & 2 & 2 \\
21 & $d \bar{d}$ & 1 & 1 \\
22 & $c \bar{c}$ & 4 & 4 \\
23 & $s \bar{s}$ & 3 & 3 \\
24 & $t \bar{t}$ & 6 & 6 \\
25 & $b \bar{b}$ & 5 & 5 \\
26 & $g g$ & 7 & 7 \\
27 & $q q g$ & - & - \\
28 & $\gamma \gamma$ & - & -\\
29 & $Z^0 \gamma$ & - & - \\ \hline
\end{tabular}
\caption{The annihilation channels \code{ch} used in \code{dswayieldone}. Also shown are the internal channel numbers \code{chi} used for the fundamental channels used in the simulations (used by routine \code{dswayieldf}). To save some additional space with the data files in memory, there are also array index channel numbers \code{chii} that are only used internally to access the right elements of the yield arrays. $S$ denotes scalars (Higgs bosons).
\label{tab:wa-channels}}
\end{table}


     For each annihilation channel and mass we simulate $10^{7}$ annihilations and tabulate the final results as a
     neutrino-yield, neutrino-to-lepton conversion rate, a muon yield and hadronic shower yields
     differential in energy and angle from the center of the Sun/Earth.
     We also tabulate the integrated yield above a given threshold and
     below an opening angle $\theta$. We assumed throughout that the
     surrounding medium is water with a density of 1.0 g/cm$^3$. Hence,
     the neutrino-to-muon conversion rates have to be multiplied by the
     density of the medium. In the muon fluxes, the density cancels out
     (to within a few percent). All results are summarized as yield tables that can be loaded and interpolated in with \ds\. This is done with the function \code{dswayieldf}. There are three kinds of yields (two-dimensional in opening angle $\theta$ and energy),
     \code{kind}=1 gives integrated yields, \code{kind}=2 gives differential yields and \code{kind}=3 gives yields integrated in angle, but differential in $\theta$. For each \code{kind}, there are 26 different \code{type}s of yield available according to Table~\ref{tab:wa-types}. As a default, only \code{type} 3--4, 9--10 and 13--14 are included in the \ds\ download, as these are the most commonly used types. If you need any other types, download the auxiliary data files from \url{http://www.darksusy.org}, unpack them in \code{share/DarkSUSY} in the \ds\ root directory, and then do a usual configure and make to install them.
Also note that the \code{kind}=3 yields are not tabulated directly, but are instead calculated and tabulated when the simulation tables are read in during \ds\ initialization.

\begin{table}
\begin{tabular}{ll}
Decay width channel &  \\
\code{dch} & Particles \\ \hline
1--29 & Same as the annihilation channels in Table~\ref{tab:wa-channels}. \\
30 & Sfermions \\
31 & Neutralinos \\
32 & Charginos \\ \hline
\end{tabular}
\caption{The neutral scalar (Higgs) decay width channels used. In \ds\, these are stored in the array \code{hdwidth(i,j)} where \code{i} is the decay channel index above and \code{j} is the Higgs number (1--3 for $H_1^0$, $H_2^0$ and $H_3^0$ respectively). For the \code{wa} routines, these decay branching ratios (partial width divided by total width) are stored in \code{dswas0br(i,j)}.
 \label{tab:hdecay0}}
\end{table}

\begin{table}
\begin{tabular}{ll}
Decay width channel &  \\
\code{dch} & Particles \\ \hline
1 & $u \bar{d}$ \\
2 & $u \bar{s}$ \\
3 & $u \bar{b}$ \\
4 & $c \bar{d}$ \\
5 & $c \bar{s}$ \\
6 & $c \bar{b}$ \\
7 & $t \bar{d}$ \\
8 & $t \bar{s}$ \\
9 & $t \bar{b}$ \\
10 & $\nu_e e^+$ \\
11 & $\nu_\mu \mu^+$ \\
12 & $\nu_\tau \tau^+$ \\
13 & $W^+ S_1^0$ \\
14 & $W^+ S_2^0$ \\
15 & $W^+ S_3^0$ \\
20 & Sfermions \\
21 & Neutralinos and charginos\\
\end{tabular}
\caption{The (positively) charged scalar (Higgs) decay width channels used. In \ds\, these are stored in the array \code{hdwidth(i,4)} where \code{i} is the decay channel index above. For the \code{wa} routines, these decay branching ratios (partial width divided by total width) are stored in \code{dswascbr(i)}.
 \label{tab:hdecay+}}
\end{table}


\begin{table}
\begin{tabular}{lll}
Yield type & & \\
\code{type} & Yield & Unit \\ \hline
1 & $\nu_e$ & $10^{-30}$ m$^{-2}$ annihilation$^{-1}$ \\
2 & $\bar{\nu}_e$ & $10^{-30}$ m$^{-2}$ annihilation$^{-1}$ \\
3 & $\nu_\mu$ & $10^{-30}$ m$^{-2}$ annihilation$^{-1}$ \\
4 & $\bar{\nu}_\mu$ & $10^{-30}$ m$^{-2}$ annihilation$^{-1}$ \\
5 & $\nu_\tau$ & $10^{-30}$ m$^{-2}$ annihilation$^{-1}$ \\
6 & $\bar{\nu}_\tau$ & $10^{-30}$ m$^{-2}$ annihilation$^{-1}$ \\
7 & $e^-$ at neutrino-nucleon vertex & $10^{-30}$ m$^{-3}$ annihilation$^{-1}$ \\
8 & $e^+$ at neutrino-nucleon vertex & $10^{-30}$ m$^{-3}$ annihilation$^{-1}$ \\
9 & $\mu^-$ at neutrino-nucleon vertex & $10^{-30}$ m$^{-3}$ annihilation$^{-1}$ \\
10 & $\mu^+$ at neutrino-nucleon vertex & $10^{-30}$ m$^{-3}$ annihilation$^{-1}$ \\
11 & $\tau^-$ at neutrino-nucleon vertex & $10^{-30}$ m$^{-3}$ annihilation$^{-1}$ \\
12 & $\tau^+$ at neutrino-nucleon vertex & $10^{-30}$ m$^{-3}$ annihilation$^{-1}$ \\
13 & $\mu^-$ at an imaginary plane at detector & $10^{-30}$ m$^{-2}$ annihilation$^{-1}$ \\
14 & $\mu^+$ at an imaginary plane at detector & $10^{-30}$ m$^{-2}$ annihilation$^{-1}$ \\
15 & hadronic shower from $\nu_e$ CC int.\ at neutrino-nucleon vertex & $10^{-30}$ m$^{-3}$ annihilation$^{-1}$ \\
16 & hadronic shower from $\bar{\nu}_e$ CC int.\ at neutrino-nucleon vertex & $10^{-30}$ m$^{-3}$ annihilation$^{-1}$ \\
17 & hadronic shower from $\nu_\mu$ CC int.\ at neutrino-nucleon vertex & $10^{-30}$ m$^{-3}$ annihilation$^{-1}$ \\
18 & hadronic shower from $\bar{\nu}_\mu$ CC int.\ at neutrino-nucleon vertex & $10^{-30}$ m$^{-3}$ annihilation$^{-1}$ \\
19 & hadronic shower from $\nu_\tau$ CC int.\ at neutrino-nucleon vertex & $10^{-30}$ m$^{-3}$ annihilation$^{-1}$ \\
20 & hadronic shower from $\bar{\nu}_\tau$ CC int.\ at neutrino-nucleon vertex & $10^{-30}$ m$^{-3}$ annihilation$^{-1}$ \\
21 & hadronic shower from $\nu_e$ NC int.\ at neutrino-nucleon vertex & $10^{-30}$ m$^{-3}$ annihilation$^{-1}$ \\
22 & hadronic shower from $\bar{\nu}_e$ NC int.\ at neutrino-nucleon vertex & $10^{-30}$ m$^{-3}$ annihilation$^{-1}$ \\
23 & hadronic shower from $\nu_\mu$ NC int.\ at neutrino-nucleon vertex & $10^{-30}$ m$^{-3}$ annihilation$^{-1}$ \\
24 & hadronic shower from $\bar{\nu}_\mu$ NC int.\ at neutrino-nucleon vertex & $10^{-30}$ m$^{-3}$ annihilation$^{-1}$ \\
25 & hadronic shower from $\nu_\tau$ NC int.\ at neutrino-nucleon vertex & $10^{-30}$ m$^{-3}$ annihilation$^{-1}$ \\
26 & hadronic shower from $\bar{\nu}_\tau$ NC int.\ at neutrino-nucleon vertex & $10^{-30}$ m$^{-3}$ annihilation$^{-1}$ \\
\end{tabular}
\caption{The yield types available from the \code{wa} routines. All of these yields are at the detector (currently IceCube). Note that the units are for integrated yields (\code{kind}=1), for differential
yields (\code{kind}=2), the units should be multiplied by GeV$^{-1}$ degree$^{-1}$.
CC int.\ = charged current interactions. NC int.\ = neutral current interactions.
\label{tab:wa-types}}
\end{table}

With these simulations, we can calculate the yield for any of these
particles for a given MSSM model.  For the Higgs bosons, which decay
in flight, an integration over the angle of the decay products with
respect to the direction of the Higgs boson is performed.  Given the
branching ratios for different annihilation channels it is then
straightforward to compute the yield above any given energy
threshold and within any angular region around the Sun or the center
of the Earth. The routine \code{dswayieldone} calculates the yield for one channel, i.e.\ even these complex channels containing Higgs bosons, whereas the main routine \code{dswayield} calculates the total
yield for a given model. Note that the WIMP annihilation yield routines do not know about SUSY at all, so before they are called, a routine \code{dswasetup} is called to set up the annihilation branching ratios for the WIMP and decay channels for the Higgs bosons. In Tables~\ref{tab:hdecay0} and \ref{tab:hdecay+}, the Higgs decay width channels are given.
 If these routines are used with other particle physics models, replace \code{dswasetup} with a routine appropriate for your particle physics model and then call \code{dswayield} as usual.

