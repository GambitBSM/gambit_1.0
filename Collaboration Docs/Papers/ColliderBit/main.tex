\documentclass[11pt,a4paper]{article}
% Presumably we will have a modified template header for the final papers
% For now we hard code things below

\usepackage{fancyhdr}
\usepackage{parskip}
\usepackage[usenames,dvipsnames]{color}
\usepackage[colorlinks=true, linkcolor=BrickRed, citecolor=Blue, urlcolor=Blue, filecolor=Blue]{hyperref}
\usepackage{graphicx}
\usepackage{xcolor}
\usepackage{amssymb, amsmath}
%\pagestyle{fancy}

\newcommand{\mail}[1]{\href{mailto:#1}{#1}}

\title{ColliderBit: A GAMBIT module for the calculation of high energy collider observables}

\author{The GAMBIT Collaboration: Alphabetical author list} % will presumably have a standard author list

\begin{document}

\maketitle

\begin{abstract}
We present a new code for the calculation of high energy collider observables, given a generic theory of beyond the Standard Model physics. Describe novel features of ColliderBit (parallelised MC generation, fast SUSY cross-section calculations, generic interface to BSM models, lots of LHC analyses). Mention link to GAMBIT framework, but emphasise that the package presents a standalone solution to the problem of applying LHC constraints to new physics theories.
\end{abstract}

\section{Introduction}
Wealth of recent LHC data provides strong constraints on new physics models. It remains difficult to apply LHC constraints to generic physics models in a rigorous fashion. Various attempts have been made to automate the process (SmodelS, CheckMATE, etc). No attempt has attacked the problem in a fully rigorous way, and neither have any existing attempts been integrated into a fully general framework for statistical fits of generic BSM theories. 

We present a new, fully general solution based on parallelised Monte Carlo simulation, fast detector simulation and fast cross-section calculations for SUSY processes. Part of the Global and Module BSM Inference Tool, which will include, etc. 

This paper is organised as follows. In Section ?? we..., etc. 


\section{Quick Start Guide}
\input{quickstartguide}

\section{Physics background}
Describe the basic design elements of the package, i.e. what we calculate given a model of BSM physics, but with the full details of the calculation deferred to the following section. Need to emphasise that the user may pick and choose each element of the calculation.

\begin{itemize}
\item Number of signal events for cut and count analyses (or in a particular bin of a binned likelihood fit) is $\sigma \times A \times \epsilon \times \mathcal{L}$.
\item State very briefly how each element of the calculation above is performed in ColliderBit.
\item Define likelihoods used in LHC searches based on the signal yields.
\item Describe LEP limit treatment
\end{itemize}


\section{User interface}
\textbf{Abram to write the following:}

\begin{itemize}
\item \textbf{Describe interface to GAMBIT framework.}
\item \textbf{Describe standalone interface (how to configure and run the package, what the defaults are).}
\item \textbf{Describe the interface to the models that we include and how to change the input parameters of those models.}
\item \textbf{Explain how to run code in single core or multicore form (i.e using the standard OpenMP commands).}
\item \textbf{Describe parameters that can be varied in yaml file.}
\end{itemize}


\section{Code description}
\textbf{For each part of the calculation outlined above: describe the code that exists within GAMBIT and/or the ready-made interface to a standard HEP code. Show validation plots (with possible links to appendices).}.

Assuming that ColliderBit is supplied a parameter point of a physics model (either standalone via an SLHA file, or from a full GAMBIT run), the basic output is a series of likelihood terms derived from the LEP and LHC BSM searches, and the LEP, Tevatron and LHC Higgs searches. The terms may then be combined to form a composite likelihood as required by the user. We here describe the strategy for calculating each individual term, along with the code implementation. 

\subsection{LHC likelihood calculation}
\subsubsection{Physics background}
A parameter point of a specified BSM model (hereafter referred to as a ``model'') can in principle be expected to show up in a variety of LHC BSM searches. Assuming that these are ``cut and count'' style analyses, the relevant data to model are the number of events that pass the kinematic cuts imposed by each analysis. If a model predicts that $s$ signal events pass the cuts for a given signal region, with $b$ expected SM background events and $o$ observed events, the likelihood of the model is given by the standard Poisson formula:

\begin{equation}
\mathcal{L}=\frac{e^{-(s+b)}(s+b)^o}{o!}
\end{equation}

Note that this does not include the effects of systematic uncertainties in the signal and background yields, $\sigma_s$ and $\sigma_b$ respectively. Assuming a distribution for the errors, one may treat them as nuisance parameters and marginalise over them as detailed in Section~\ref{sec:stats}.

LHC BSM search papers provide details of $b$ and $o$ for each signal region, along with the background uncertainty (and some estimate of the signal uncertainty for representative models). Calculating the likelihood for a given model thus requires an accurate estimate of $s$ which is given by:

\begin{equation}
\label{eq:s}
s=\sigma \times A \times \epsilon \times L
\end{equation}

where $\sigma$ is the production cross-section for processes for which the detector has an acceptance of $A$ and an efficiency of $\epsilon$. $L$ is the integrated luminosity of data used in the search (typically 20.1 fb$^{-1}$ for the searches considered here).

The only fool proof way to calculate $s$ is to perform a cross-section calculation at the desired accuracy before evaluating the acceptance and efficiency via a Monte Carlo simulation of the LHC collisions followed by a simulation of the ATLAS and/or CMS detectors. One can then apply the analysis cuts for a given LHC search to the results of the detector simulation. An approach using efficiency look up tables and extrapolation from simplified models removes the need for simulation, but gives very conservative results as a consequence, leading to an underestimate of the LHC reach. Furthermore, the generation of look up tables must be repeated for every physics model of interest, making it hard to produce a generic code for the application of LHC constraints.

The overriding strategy of the \tt ColliderBit \rm LHC module is instead to make each step of the simulation chain faster using a combination of custom speed increases and parallel computing. The package thus performs a cross-section calculation, generates Monte Carlo events, performs an LHC detector simulation and then applies the analysis cuts for a range of LHC analyses using a custom event analysis framework. The user can then utilise the GAMBIT statistical routines to return LHC likelihoods. The basic processing chain is illustrated in Figure~\ref{fig:lhcchain}, and the code is designed so that the user can freely choose which software performs each step of this process. Nevertheless, \tt ColliderBit \rm has a default chain implemented, and the first release contains the elements summarised in the following subsections.

\begin{figure}[ht]
\centering
\includegraphics[width=0.8\textwidth]{figures/Gambit-LHC-Chain.pdf}
\caption{Schematic diagram of the \tt ColliderBit \rm processing chain for LHC likelihoods.\label{fig:lhcchain}}
\end{figure}

\subsubsection{Cross-section calculations}

We use the LO+LL cross-sections calculated numerically by the popular \tt Pythia 8 \rm event generator. For many models, these are state of the art. For models where a NLO (or better) calculation is available (e.g. SUSY) our results are conservative, but considerably quicker to evaluate than the full NLO results obtained using the standard \tt Prospino \rm tool~\cite{Beenakker:1996ed}. 

\textbf{For the first release, we should also provide an easy method of having a user-supplied cross-section, perhaps by adding a module function that reads the cross-section from a text file.}

\subsubsection{Monte Carlo event generation}

We supply an interface to the \tt Pythia 8 \rm event generator, alongside custom code that parallelises the main event loop of \tt Pythia \rm using \tt OpenMP \rm (for a similar approach, see~\cite{Lester:2005je}). This allows generation of $\approx$ 10,000 SUSY events on an Intel Core i7 processor in less than 10s (\textbf{Should we note the necessary compile options here?}). In addition, in a GAMBIT-driven global fit, points are vetoed if the production cross-section is already too low to lead to observable consequences at the LHC, giving an implicit speed increase. Taken together, these routines make the running of a full Monte Carlo simulation in a global fit computationally tractable. The choice of the \tt Pythia \rm generator is an acceptable compromise between generality and ease of use for the first \tt ColliderBit \rm release. It is sufficient for many BSM models, and is easily extendable with matrix elements for new models via the existing \tt Madgraph 5 \rm interface (for an example, see Section ADD). \tt Pythia \rm will prove insufficient, however.  where NLO diagrams are significant (for example, in the accurate treatment of some effective field theories of dark matter in LHC monojet searches, where top quark loops become important~\cite{Buckley:2014fba}). These deficiencies can be fixed in the current release via a user-supplied interface to an appropriate Monte Carlo tool, and such interfaces will be supplied as standard in future \tt ColliderBit \rm releases.

\textbf{Should we provide instructions for how to interface to a new MC tool for the first release? Perhaps we should to make sure that ColliderBit doesn't fall over.}

\subsubsection{Detector simulation}
\tt ColliderBit \rm is structured so that the detector simulation is run during the parallelised main event loop, thus giving an implicit speed increase of the simulation step. The user has several options for this step, and in all cases jets are reconstructed usiung the \tt Fastjet \rm package~\cite{Cacciari:2011ma}.

\begin{center}
\emph{No detector simulation}\\
\end{center}

The user can choose to run no detector simulation, in which case the truth level MC events are passed directly to the event analysis framework. Jets may be defined directly at the parton level, or at the hadron level. The former is only really sufficient for analyses in which leptons are the main species of interest, in which case turning off hadronisation can lead to a large speed increase.

\begin{center}
\emph{Delphes}\\
\end{center}

We provide an interface to the \tt DELPHES \rm detector simulation (\textbf{Which version?})~\cite{deFavereau:2013fsa,Ovyn:2009tx} which provides respected simulations of the ATLAS and CMS detectors. DELPHES includes a simulation of track propagation in the magnetic field of an LHC detector, along with a simulation of the electron and hadron calorimeters, and the muon chambers. The user can configure the parameters of the simulation using the normal DELPHES card, but it should be noted that $b$ tagging and the ATLAS lepton ID selection efficiencies (``medium'',''tight''), etc, are applied within the event analysis framework in order to allow different analyses to use different working points.

\begin{center}
\emph{BuckFast}\\
\end{center}

For many purposes, a more approximate (and much faster) approach based on four vector smearing is sufficient, and we therefore also supply an internal \tt ColliderBit \rm detector simulation, \tt BuckFast \rm, that uses a mixture of the DELPHES momentum and energy resolution functions and new parameterisations of the ATLAS and CMS identification efficiencies, as follows: (\textbf{Need to add CMS functions for first release. Also should we update the DELPHES functions used for the initial efficiency and resolution?}.  

\begin{itemize}
\item \textbf{Electrons: }We apply, to the truth level electron four vectors, the DELPHES functions for electron tracking efficiency, electron energy resolution and electron reconstruction efficiency (in that order). In the analysis step, we apply parameterisations of the ATLAS medium and tight identification efficiencies (as appropriate), taken from~\cite{ATLAS-CONF-2014-032}.
\item \textbf{Muons: }We apply, to the truth level muon four vectors, the DELPHES functions for the muon tracking efficiency, the muon momentum resolution and the muon reconstruction efficiency (in that order). 
\item \textbf{Taus: }Hadronic taus are identified at truth level, and selected using a flat 40\% efficiency. The energy is smeared by a 3\% jet energy resolution.
\item \textbf{Jets: }Jets are reconstructed at hadron level using the anti-kT algorithm of the \tt Fastjet \rm package \textbf{Need to add configurable jet radius.} The energy is smeared by a 3\% jet energy resolution.
\item \textbf{Missing energy: }\textbf{Andy? I can't remember exactly how we do this.}
\end{itemize}

Andy to add a selection of DELPHES vs BuckFast validation plots for a representative signal process. Should also add some cutflow comparisons (MJW)?

\subsubsection{Event analysis framework}

\begin{itemize}
\item Describe event analysis framework
\item Emphasise applicability to simulated and/or reconstructed events
\item Describe the LHC analyses included in the module and include some validation plots showing some nicely reproduced ATLAS and CMS limits.
\item Explain how to add new analyses
\end{itemize}

\subsubsection{Statistics}
\label{sec:stats}
\begin{itemize}
\item \textbf{Pat to describe the likelihoods implemented in ColliderBit.cpp (i.e. quote formulae for log normal and Gaussian error distributions).}
\end{itemize}
Having obtained a likelihood for each CMS and ATLAS signal region, we calculate the overall LHC likelihood by using the observed likelihood obtained using the signal region that gives the best expected limit. This is conservative, but is the appropriate treatment when one lacks sufficient information to handle correlated systematic uncertainties.

\subsection{LEP likelihood calculation}
\textbf{Are to write short summary.}

\subsection{Higgs likelihood calculation}
\textbf{Chris Rogan to write short summary.}



\section{Examples}
\textbf{Give some tutorial-style examples for common use cases.}


\subsection{MSSM example}

\begin{itemize}
\item Reproduction of an LHC CMSSM exclusion limit?
\item How to scan over pMSSM parameters and return a likelihood.

\end{itemize}


\subsection{Generic Pythia model example}
Give an example of Madgraph matrix element code to Pythia 8, show how to get it working and how to scan over the parameters and return a likelihood. Might be ambitious for the first paper, but it wouldn't take too long and would be a superb feature for increasing the user base.



\section{Conclusions}

\section{Acknowledgements}

\section{Appendix: Validation (optional)}

\end{document}
